\documentclass[11pt]{article}
\usepackage{fullpage}


\title{Your title} 
\author{Your name}

\begin{document}

\maketitle

\section{Basic LaTex}

The basic things you need to know are math mode.  Inline math mode is given by dollar
signs, e.g., let $a(n) = \sum_{n=1}^r f(r)$.  And set off equations are given by
\[ x^2 + Dy^2 = n \]
(unnumbered)
or
\begin{equation}
  \alpha \bar \alpha = (1+\sqrt{-5})(1-\sqrt{-5}) = 2 \cdot 3 = 6
\end{equation}
(numbered).

Can reference bibliography with cite.  E.g., In \cite{Hej}, it was shown that \dots.
Typically, you have to typset twice to get the reference references correct.

You can also make subsections

\subsection{Subsection}

and do {\bf bold} and {\em italics}.

\medskip
\noindent
There are many latex references on the web, so you should have no trouble finding other
things you need to know.  If you have trouble, ask me.

\section*{References}

\begin{enumerate}

\bibitem[H]{Hej} Hejhal, Dennis A. The Selberg trace formula for ${\rm PSL}(2,R)$. Vol. I. Lecture Notes in Mathematics, Vol. 548. {\em Springer-Verlag, Berlin-New York,} 1976.

\end{enumerate}

\end{document}
